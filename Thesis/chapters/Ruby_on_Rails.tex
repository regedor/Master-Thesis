\thispagestyle{empty}
\chapter{Ruby on Rails}\label{chap:ruby_on_rails}


Ruby é uma linguagem, Rails é um framework

\section{Ruby Language} 
Escrever sobre Ruby, aqui!

\section{Ruby on Rails Framework} 
Escrever sobre RoR, aqui!


%  In 2010 Richar Huang (flyerhzm) created Rails Best Practices (url), 
%  inspired by a talk given by Wen-Tien Chang (ihower) at 2009 Kungfu RailsConf in Shanghai China 
%  (http://www.slideshare.net/ihower/rails-best-practices).
% 
% O site tem como administradores estes dois e mais dois colaboradores.
% O objectivo é discutir e chegar a consenso sobre boas praticas a seguir quando são criadas aplicações rails.
% 
% The web site contains two main sections:
% 
% 1. Best Practices
%   All registered user can suggest best practices.
%   Best practices are voted and commented by users.
% 
%   Qualquer utilizador registado pode sugerir uma boa pratica.
%   Existe um sistema de votação constante
%   E é possível na pagina dessa boa pratica discutir a mesma sob a forma de comentários
%   O site tem cerca de 70 boas praticas a maioria sugerida pelo 2 autores
%   Parece-me que a comunidade de votantes nas varias praticas é cerca de vinte pessoas
%   Mas muitas destas paginas de boas praticas já foram visitadas por mais de 5000 pessoas
%   Em parte estas pessoas devem advir da utilização criada pelo autor que valida automaticamente 
%   as praticas no aplicação e apresenta um relatório com link para a pagina onde se da a discussão.
%   Esta gem apenas tem implementadas 25 das 70 boas praticas.
%  
% Alem disto existe uma serie de gems para fazer diversas analises ao codigo, 
% e até uma que agrega estas. Existe um video aqui que mostra estas em funcionamento:
% http://media.railscasts.com/videos/252_metrics_metrics_metrics.mov
% 
% 2. Secção de questões:
%   O site tem ainda uma área de perguntas e respostas. Tipo forum de discução.
% 
% 
% Outro ponto interessante é que é mais fácil encontrar projectos ruby open source 
% (por exemplo o próprio rails, ou gems que tratam diversos problemas), do que applicações rails.
% Embora na realidade, provavelmente, existam mais aplicações rails do que projectos em ruby 
% (claro que estas se podem incluir nos projectos ruby) isto acontece porque as aplicações rails são o produto final, 
% fazem uso dos projectos ruby, mas na maior parte do caso não são reutilizadas.

%% RUBY
\section{Best Practices in RoR Projects} \label{sec:ror_best_practives}
Ruby is a dynamic, object oriented, open source programming language created by Yukihiro Matsumoto and public released in 1995.
It has an elegant syntax that is natural to read and easy to write.
Ruby has drawn devoted coders worldwide. In 2006, Ruby achieved mass acceptance.
Moreover, the web framework Ruby on Rails is considered the biggest responsible for Ruby popularity
( tens of thousands of Rails applications are online).
%I conclude that the biggest part of ruby developers are in fact Ruby on Rails or Web developers, they tend to write HTML, javascript, CSS and use ruby gems maintained by small rubists communities (Ruby on rails is in fact a compilation of small projects). Rails developers write ruby code inside their rails app, but in many cases they have never wrote a ruby gem.
%There are a lot of frameworks and projects but I think that the only framework that deserves a big study is Ruby on Rails, all other frameworks have small communities compared to it. The other possibility is to find out best practices of ruby code in general, thinking of it as script code and not with a project structure.

Ruby and Ruby on Rails community members are, in general, addicted to best practices.
However, in reality, many of those best practices are studied development methodologies.
For instance, the majority of Ruby on Rails book authors speak about automated tests, written using specific DSLs, like Cucumber or Rspec.
It is also common to associate Ruby on Rails with Behaviour Driven Development (BDD) and Agile methodologies.

Because of all this, the ruby community has great potential to be a starting point to understand the role of best practices, its benefits and how to measure it.
In fact, there is already some work done.

The web site
\textsf{Rails Best Practices}\footnote{\url{http://www.rails-bestpractices.com/} is a web site created by Richard Huang,
it was inspired by Wen-Tien Chang talk given at Kungfu RailsConf 2009 in Shanghai. Slides can be found here
\url{http://www.slideshare.net/ihower/rails-best-practices}.},
works in similar way to a web forum and its objective is to engage developers to discuss which practices
should be considered best practices to follow, when building a RoR web application.
The community involved with this web site is committed to build a
gem\footnote{Ruby Libraries are called gems. Ruby gems can be easily managed using rubygems (rubygems is for Ruby as aptitude is for Debian or cpan for perl).}
that produces a report about a given project.


\subsection{Ruby Best Practices Examples}
But what is is a best practice after all?
Best practices can be related to code formatting:
\begin{itemize}
\item \emph{Use two spaces to indent code and no tabs}, it is a matter of taste but every worthy ruby developer do it that way.
\item \emph{Remove trailing whitespace}, trailing whitespace makes noises in version control systems.
\end{itemize}

Can be related to syntax:
\begin{itemize}
\item \emph{Avoid return where not required}.
\item \emph{Suppress superfluous parentheses}, when calling methods, but keep them when calling \"functions\" (when you use the return value in the
  same line).
\end{itemize}

Can be related to naming:
\begin{itemize}
\item \emph{Use snake\_case for methods}.
\item \emph{Other method naming conventions}: Use map over collect, find over detect, find\_all over select, size over length.
\end{itemize}

And can also be specific to a framework, rails best practices:
\begin{itemize}
\item \emph{Law of Demeter}, A model should only talk to its immediate association.
\item \emph{Move code into controller}, according to MVC architecture, there should not be logic codes in view.
\item \emph{Isolate seed data}, do not insert seed data during migrations, a 
rake task\footnote{ Rakefiles work in similar way to Makefiles but are written in ruby. It is a simple way to write code to automate repetitive tasks. }
can be used instead.
\item \emph{Do not use default route}, When using a RESTful design. The default RoR routes can cause a security problems.
\item \emph{Replace Complex Creation with Factory Method}, Sometimes you will build a complex model with params, current\_user and other logics in controller, but it makes your controller too big, you should move them into model with a factory method.
\end{itemize}
