\thispagestyle{empty}
\chapter{Introduction}\label{chap:introduction}


Nowadays, Open Source Software (OSS) is well disseminated.
Thousands of OSS packages can be found online, and free to download,
in Open Source Project Hosting Websites (OSPHW) like
\textsf{SourceForge}\footnote{\url{http://sourceforge.net/}.},
\textsf{Google Code}\footnote{\url{http://code.google.com/}.}, or
\textsf{GitHub     }\footnote{\url{https://github.com/}.}.
Those websites, usually in conjunction with a Version Control System (VCS), make it easy for developers, all around the globe,
to collaborate in Open Source Software Projects (OSSP), and also act as a way to make software available to users.

%Nevertheless, OSPHW are not the only prof of OSS establishment.
According to \textsf{NetCraft}\footnote{\url{http://news.netcraft.com/archives/2010/05/14/may\_2010\_web\_server\_survey.html/}, accessed on 2010/12/21.},
the market share for top servers across the million busiest sites was 66.82\% for the open source web server, Apache,
much higher than the 16.87\% for Microsoft web servers in May 2010.
Even governments started noticing open source, during the last few years, and in some case adopted it\cite{hahn2002government}.
The broad acceptance of OSS means that now OSS is not only used by computer specialists.

\textsf{John Powell}\footnote{John Powell is CEO, President, and Co-founder, Alfresco Software Inc.}
has declared that measuring the savings that people are making in license fees, the open-source industry is worth 60 billion dollars.
\textsf{Matt Asay}\footnote{Matt Asay is chief operating officer at Canonical, the company behind the Ubuntu Linux operating system.}
shares the view that from the customers perspective open source can be now considered the largest software industry in the world.
The full review can be found at \textsf{CNET News}\footnote{\url{http://news.cnet.com/8301-13505\_3-9944923-16.html/} accessed on 2010/12/21.}.

Usually large industries have a strict organization model, that is not the way open source communities operates.
Open Source communities work in a kind of \textit{bazaar style}.
~\cite{raymondcathedral} compares the traditional software development process to built cathedrals,
few specialized individuals working in isolation.
While open source development seemed to resemble a great babbling \textit{bazaar}.
But OSS is not developed, all the time, in \textit{bazaar style} and each community can have particular habits.
Currently, big open source projects can have companies supporting them.
However, most projects are not that big and sometimes it is hard to distinguish the project developers from the project customers/users,
because of that bug reports and wanted features can get indistinguishable too.
The specification of an open source software project evolves in an organic way~\cite{capiluppicathedral}.

Can software that is developed in such chaotic way be trusted as a high quality product?
The shock is that in fact the \textit{bazaar style} seemed to work~\cite{halloran2002high}.
Some big projects, for instance Linux distributions such as \textsf{Ubuntu}\footnote{\url{http://www.ubuntu.com}.
Ubuntu is a free \& open source operating system.},
are the proof of it.
However, how can the quality of this software be measured?

The most basic meaning of software quality is commonly recognized as lack of "bugs", and the meeting of the functional requirements.
But quality is not simply based on that~\cite{gousios2007software}.
The quality of a software system depends, among other things, on update frequency, quantity of documentation, test coverage,
number and type of its dependencies and good programming practices.
By analysing those parameters a user can make a better choice when picking software for a specific task~\cite{marchenko2007predicting}.

When a user/developer finds a new OSSP, for example in
GitHub, the things that will most influence the time needed to have a better understanding of the project, to use, or collaborate in it,
are the quality of the documentation and the source code readability.
Although the OSPHWs provide plenty of useful information about the hosted projects,
currently, they do not give a quick answer to the following questions:
Does this project have good documentation? Does the code follow standards? How similar is it to other projects?

An OSSP is built up from hundreds, sometimes thousands, of files. It can be coded in many different computer languages.
To  analyze manually a software project is a very hard and time consuming task,
and not all users have the ability to answer the previous questions by looking at the source code~\cite{crowston2003defining}.

However, open source communities are constantly creating and improving their working methodologies.
And even without noticing, communities create rules and best practices.
By following those \emph{best practices}, software projects increase their maintainability level.

With that in mind, a system capable of analyzing and measuring a given OSSP,
producing detailed quantitative and qualitative reports about it,
would enable users to make better choices and, of course, developers to further improve the package.

This paper discusses the concept of Quality when addressing an OSSP, and how to measure it (Section 2) using classic approaches.
After that,  the notion of \emph{best practices} is introduced and  the impact  of
taking their use into account when assessing an  OSSPs is explored.
To make this proposal clearer and stronger,
\textsf{Ruby}\footnote{Ruby is an open source programming language. Ruby community is relatively young but still very focused on following best practices.  }
community is taken as a starting target (Section 3).
At last but not least, to support our proposal a case-study is shown, in Section 4: seven Ruby OSSP are measured and compared.