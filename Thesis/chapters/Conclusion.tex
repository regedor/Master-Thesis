\thispagestyle{empty}
\chapter{Conclusion}\label{chap:conclusion}


As it was said in the beginning of this document, thousands of open source software packages can be found online and 
freely downloaded in platforms like GitHub,
the web-based hosting service for projects using the Git revision control system that 
hosts more than 1 million open source projects.

Throughout this study the need of assessing the quality of this open source software projects became clear.
Maintainability turned out to be a crucial attribute to assess their quality, 
this happens obviously because of the community driven orientation of this kind projects, 
described in chapter \ref{chap:open_source}. 
However, this statement does not mean that maintainability is not also a quality attribute of great importance
for closed source software projects too. 
In fact most standards, studies and tools within the field of measuring the quality of software projects 
were conducted by private companies with the purpose of improving and assuring the quality of their projects,
usually closed source software projects, this idea together with the concepts of software quality and 
quality attributes and was explored in chapter \ref{chap:software_metrics}.

Few people have the ability to quickly assess the quality of a project by looking at source code. 
An application, able to perform automatic analysis of a software projects and to generate a hight level overview of the code,
is beneficial for the quality of general software projects, independently of the license, open source or not.

Nevertheless, open source and closed source usually end up having a different development process, 
because of that they should obviously be analyzed in different ways.
In the open source world there are no strict development rules, 
but as time goes by and projects get bigger something similar to rules
or standard behaviors start to emerge in an spontaneous way, we call this best practices.
There is the belief that by following this best practices better results will be achieved.

In fact, the existing of common guide lines in software projects intuitively seems to be potenciator of its 
maintainability to every developer.
But just like popular sayings, their value is ambiguous if there is no scientific prof and 
the usage could be considered a matter of taste.

However, we strongly believed that there is a big value relies in the concept of best practices.
Best practices are not rigid as standards this means that their usage is also more flexible.
Despite this, our belief since the beginning of this investigation is that if 
a project follows best practices, it has a higher probably of being a better quality project.
Following this idea we focused our research in proving the existence of
a relation between the quality of the GitHub open source projects and the amount of best practices followed.

In the beginning of the writing of this document looked like there was no work done 
concerning the measurement of best practices by automatic analyzing source code.
As a good surprise and backing up this ideas described in this document, at the same time this research was conducted, 
various projects appeared in scene based on similar ideas and concepts.

In the particular context of Ruby Community, we found that some efforts had already been taken in this exact direction,
defining best practices and create scripts to automatic analyze the source code.
A project called Rails Best Practices, started by Richard Huang, 
which managed to put in practice some of our initial ideas of defining best practices with 
the help of a specific community input and opinions, in this case rails developers,
and even started to create code analyzers to detect if a project is following them,
was described in chapter \ref{chap:best_practices}.

However, like most of the reports generated by the existent source code analyzers
the rails best practices gem implementation is only spotting the occurrences of bad smells.
Not really identntifying the best practices, but instead the not best practices. 
In chapter \ref{chap:assissing_ror} we named this concept of NBPs.
Doing it so is a cleaver idea it verify if a best practice is being followed, 
since most of the time it is easier to identify incorrection unlike correction.
 
This nbps reports are of great help to improve the quality of rails projects, but this is not enough.
There was the need to interpret those results to end up with a high level quality statement.
The expectation was that by analyzing a massive amount of open source projects, it would possible to start understanding
the meaning of the values given by simple metrics,
thus to judge the impact of the this numbers in the quality of the project. 
Consequently, find new ways to assess the quality, close related with the level of maintainability of an open source project.
The final objective was to create a new higher level metric capable of quantify the best practices followed by a given project.
Something as simple as "from 1 to 5, this project has 4 stars in terms of best practices".

To achieve this we started looking for rails projects hosted in GitHub, 
applying the different metrics and running the NBPs checkers and gathering all this information.
Like it was said before, by itself those value are useless, yet
by comparing projects, it was possible to start to understand how to interpret those values.
For instance, obvious things like considering the \emph{size of the project}
(it is intuitive that a project with 10 lines of code and 10 errors is worse than a project with 1000 and 20 errors),
made a huge difference when taken into consideration.

From the studies we carried out and, described in the paper,
we also have learned that each best practice has a different importance level
1 NBP that affects security or performance is, for sure, worse than 10 NBPs related to indentation;
or 10 NBPs related to naming conventions are worse than the indentation mistakes.
But never forgetting, that as best practices they are the importance of each one should be defined
by the community in the Rails community case, by the votes comments and reactions of users from
the rails best practices web site.

Lots of scripts were written to automate the code analyses process during the initial studies, 
and after analyzing more than 50 projects it was possible to find correlations
between the usage of best practices describe and discussed in rails best projects project and
the activity and popularity of projects in GitHub (number of projects forks, project contributors and people following updates).
Along the paper, we gave arguments in order to make evident that it is worthwhile to detect on the source code
that the author follows the best practices recommended by the respective community, 
and after finding the correlations it was undeniable.
Having proved this thesis it was straight forward to build a web application that 
given a set of projects can generate a NBPs report and also a global score, 
a score from 1 to 5 in terms of following best practices.

As future work, more correlations should be explored
Some of those variables have been already identified as of great value: the number of commiters, starting date of the project, last commit data, and total number of commits. 
Those variables are also strongly related with the forks, watchers and of course, in the end, the quality of the project.
For instance, when the last commit of the project is to old it can probably the justification for a bad score, 
since best practices are always being renewed, hardly an old project can be following it in an exemplary manner.

Some of this subtleties are already partly taken into account in the developed application.
But for now just as a way to justify the obtained score. 
Nevertheless, the are countless variables that care study, and can possibly improve this project.
Nevertheless, it was already possible to run the application against a big set of projects.
There are more than 100 projects analyzed.
A list of analyzed projects can be found at \url{app.study.gourgeouscode.com}.
The reports and the overall score can be of great help both for users when comparing projects and 
also for open source developers whiling to contribute to those amazing projects.

Software projects, societies or any other point of human interaction do need some kind of order.
Strict rules are accepted as necessary evil, the only solution, always defined from the top of the pyramid,
always cutting the freedom, but without freedom no advance is possible.
Best practices are necessary order in a free world.