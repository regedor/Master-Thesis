\thispagestyle{empty}
\chapter{Best Practices}\label{chap:best_practices}


\section{Best Practices in OSSP development} \label{sec:best_practices}
Open source communities have a tendency to create coding standards. It is a natural and evolutive process.
Standards are not rules but instead best practices that are spread through the community and everybody does it that way.
Furthermore, best practices discourage:
\begin{itemize}
\item Poor performance (due to bad patterns)
\item Poor error checking (defensive programming)
\item Inconsistent exception handling / Maintainability (long-term quality)
\end{itemize}
When a developer follows the standards and best practices, the project maintainability is increased.
Consequently, project new comers will find it easier to understand the project codebase~\cite{dromey2002model}.

However, there is little work done concerned with measuring coding standards by automatic analyzing source code.
A plausible explanation for that is the fact that best practices are not a set of immutable rules,
they are a continuous evolution and improvement of development methodologies.
Communities are constantly creating rules and best practices, even without noticing it.
It is not possible to write down a list of best practices without some ambiguities.
Nevertheless, it is still possible, to use metrics on the source code and, by analyzing their values,
to find hints to help answering weather some methodological approaches
were taken into account during the project development process.

At first glance, best practices metrics are for classic metrics as natural as medicine is for science.
But, it is not the case.
In fact, classic metrics, on their own, do not give much information about a project.
In many cases, best practices can be the key to understand what should be the optimum value for a classic metric,
for instance, how many lines of code should a ruby method have?

Of course, those questions are subjective.
However by analyzing renowned projects, developers opinions and so on, it is possible to find the best practice
and that gives a plausible answer to the optimum value.

We believe that, actually, best practices can give a meaning to metrics.