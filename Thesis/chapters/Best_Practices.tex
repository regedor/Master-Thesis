\thispagestyle{empty}
\chapter{Best Practices}\label{chap:best_practices}

Best practices are not rules, they are standards followed by a specific community.
Something that most people agrees with, although it might not be scientifically proven,
it seems to be the best way of handling a problem.

In this chapter we will understand how important best practices are for open source communities, 
how they give meaning to traditional software metrics
and identify some examples of best practices from the ruby community.


\section{Best Practices in Open Source Software Projects} \label{sec:best_practices_ossp}
Open source communities have a tendency to create \emph{coding rules},
i.e., \emph{principles governing the conduct of programmers and serving as a basis of measure or judgment}. 
It is a natural and evolutive process for people surviving in an open space.
We can call these natural rules: \emph{best practices}.

Best practices are  methods thought as being the  best way for achieving something;
they are spread through the community and everybody does it that way.
It is obvious that when a developer follows well established principles and best practices, 
the project maintainability is increased.
Consequently, project new comers will find it easier to understand the project code~\cite{dromey2002model}.
But there is more than that, following best practices discourage:
\begin{itemize}
\item Poor performance (due to bad patterns)
\item Poor error checking (defensive programming)
\item Inconsistent exception handling / Maintainability (long-term quality)
\end{itemize}

To attain the same benefits, companies define standards, this is, something considered by an authority 
as a basis of comparison and a normal requirement for quality;
by other words, an approved behavior model for their workers.
However, these principles are defined by the few people on top and then spread down on the pyramid.
Many times, those rules are not well thought by the project leaders and that can block the progress.

In the other way around, the apparent chaos of open source also requires some rules,
but contrary to the companies coding standards, which work in a top down way,
best practices happen in a bottom up and distributive mode,
everybody can try different ways of doing things,
but the ones with better results are most likely to be copied.

A simple metaphor exposes the difference:
\begin{quote}\emph{
  Companies use traffic lights where open source communities use roundabouts.
}\end{quote}

Both, the strict company standards (traffic lights) and the OSP best practices (roundabouts) are ways to regulate intersections.
The result of traffic lights is  easier to predict, however that regulation  system does not depend much on the drivers skills;
because it is so restrictive it will happen often to find a driver stooped alone
at the crossroads waiting for a green light, losing precious time.
On the other hand, the roundabout system is a less restricted system and relies much more on the quality of the drivers,
but it open the possibility to a much more efficient way to avoid a traffic jam.

There is little work done concerned with measuring coding best practices by automatic analyzing source code.
A plausible explanation for that is the fact that best practices are not a set of immutable rules,
they are a continuous evolution and improvement of development methodologies.
Communities are constantly creating rules and best practices, even without noticing it.
It is not possible to write down a list of best practices without some ambiguities.
Nevertheless, it is still possible to use source code metrics and, by analyzing their values,
to find weather some methodological approaches were taken into account during the project development process.

At first glance, best practices metrics seam to be for classic metrics as natural as medicine is for science.
But, it is not the case.
In fact, classic metrics, on their own, do not give much information about a project.
In many cases, best practices can be the key to understand what should be the optimum value for a classic metric,
for instance, to determine \emph{how many lines of code should a ruby method have}.

Of course, those questions are subjective.
However, by analyzing renowned projects, developers opinions and so on, it is possible to find out a best practice
that gives a plausible answer to the search for the \emph{most favorable value}.

We believe that, actually, best practices can give a meaning to metrics.

\section{Identifying Best Practices} \label{sec:identifying_best_practices}
We understood that software projects can benefit a lot from using best practices.
However, what is a best practice after all?
The truth is that everything can be a best practice, for example:
the use of two spaces to indent code and no tabs, 
writing unit tests for your code,
the way files are organized inside a project, etc.

Some of those things might look like a matter of taste,
but every worthy ruby developer uses two spaces to indent code.
This is the standard for the ruby community.
Other communities, for instance JavaScript programers, prefer 4 spaces
and in the Java world 8 spaces is considered the considered a good choice.

It is important to notice,
that we can find virtually no ruby programmer using a different indentation, 
but we can find Java communities advocating different indentations.

This fact shows that the ruby community agrees that 2 spaces is a the best option and
can be considered a best practice.
In contrast, we can not be completely sure about the 8 spaces for Java,
since there is also a lot of developers advocating 4 spaces and also a good number using tabs instead.
The Java community is divide, it almost possible to identify sub communities defending 
different answers for the same questions. 
It might be possible to identify conventions in those smaller communities, 
but it will be harder to do it for the bigger community.

To consider this conventions a best practice, it is important to understand if that
conventions are strong through the community in question. 

Obviously, the first step is to identify the community:
are we trying to find best practices for ruby programers general? for all programers in general? 
for the programers working in a specific company or project?
In smaller communities, it might be easier the achieve agreement and
consequently find best practices.
Nevertheless, practices created and followed by a small group of programers,
are less likely to be considered best practices in general compared to 
practices followed by the developers working on the top 10 open source projects.

It is clear that best practices are specific to a certain community, 
so, after correctly choosing a community, to find identify best practices we need
to find out what patterns are used by its members,
and try to prove that some benefit comes from using it.

The most obvious benefit from using it,
is that the project maintainability is increased,
for example: people in the community might be expecting to find the code indented with two spaces or 
methods named in camelcase, constants in upcase, etc.

Supposedly using two or more spaces does not affect directly the code quality (other than maintainability), 
but it is reasonable to infer that a developer is new to ruby if he does not know it.
In other words, following, or not, best practices has a relation with the developer knowledge and experience,
and the developer experience is likely to be related with the code quality produced.

Of course, best practices are not only related to naming,
they can also be patterns for solving certain problems, working methodologies, etc.
In those cases, they might be directly related to other quality attributes like performance and so on.

In the end, it seems plausible to believe that there is a correlation between this two variables.
The quantity of best practices followed and the overall quality of the project. 

Later in this document, this relation is proved.


\section{Ruby Best Practices Examples} \label{sec:best_practices_examples}
In this section, different best practices categories are listed.
These are best practices for the ruby community.

Best practices related to code formatting:
\begin{itemize}
\item \emph{Use two spaces to indent code and no tabs}, every worthy ruby developer do it that way.
\item \emph{Remove trailing whitespace}, trailing whitespace makes noises in version control systems.
\end{itemize}

Related to syntax:
\begin{itemize}
\item \emph{Avoid return where not required}.
\item \emph{Suppress superfluous parentheses}, when calling methods, 
but keep them when calling \"functions\" (when you use the return value in the same line).
\end{itemize}

Related to naming:
\begin{itemize}
\item \emph{Use snake\_case for methods}.
\item \emph{Other method naming conventions}: Use map over collect, find over detect, find\_all over select, size over length.
\end{itemize}

Specific to a framework (Ruby on Rails in for the following examples):
\begin{itemize}
\item \emph{Law of Demeter}, A model should only talk to its immediate association.
\item \emph{Move code into controller}, according to MVC architecture, there should not be logic codes in view.
\item \emph{Isolate seed data}, do not insert seed data during migrations, a 
rake task\footnote{ 
  Rakefiles work in similar way to Makefiles but are written in ruby. It is a simple way to write code to automate repetitive tasks. 
} can be used instead.
\item \emph{Do not use default route}, When using a RESTful design. The default RoR routes can cause a security problems.
\item \emph{Replace Complex Creation with Factory Method}, Sometimes you will build a complex model with params, current\_user and other logics in controller, but it makes your controller too big, you should move them into model with a factory method.
\end{itemize}



%% RUBY
\section{Ruby on Rails Best Practices} \label{sec:ror_best_practives}

Ruby and Ruby on Rails community members are, in general, known has been addicted to best practices.
In reality, many of those best practices are studied development methodologies.
It is also common to associate Ruby on Rails with Behaviour Driven Development (BDD) and Agile methodologies.
Most developers, even language new commers, are worried about writing automated tests, and following best practices.
and there is a lot of interesting domain specific languages for it, like Cucumber or Rspec.


For instance, the majority of Ruby on Rails book authors speak convention over configuration,
most those conventions can be also called best practices.


Because of all this, the ruby community has great potential to be a starting point to understand the role of best practices, its benefits and how to measure it.
In fact, there is already some work done.

The web site
\textsf{Rails Best Practices}\footnote{\url{http://www.rails-bestpractices.com/} is a web site created by Richard Huang,
it was inspired by Wen-Tien Chang talk given at Kungfu RailsConf 2009 in Shanghai. Slides can be found here
\url{http://www.slideshare.net/ihower/rails-best-practices}.},
works in similar way to a web forum and its objective is to engage developers to discuss which practices
should be considered best practices to follow, when building a RoR web application.
The community involved with this web site is committed to build a
gem\footnote{Ruby Libraries are called gems. Ruby gems can be easily managed using rubygems (rubygems is for Ruby as aptitude is for Debian or cpan for perl).}
that produces a report about a given project.




%  In 2010 Richar Huang (flyerhzm) created Rails Best Practices (url), 
%  inspired by a talk given by Wen-Tien Chang (ihower) at 2009 Kungfu RailsConf in Shanghai China 
%  (http://www.slideshare.net/ihower/rails-best-practices).
% 
% O site tem como administradores estes dois e mais dois colaboradores.
% O objectivo é discutir e chegar a consenso sobre boas praticas a seguir quando são criadas aplicações rails.
% 
% The web site contains two main sections:
% 
% 1. Best Practices
%   All registered user can suggest best practices.
%   Best practices are voted and commented by users.
% 
%   Qualquer utilizador registado pode sugerir uma boa pratica.
%   Existe um sistema de votação constante
%   E é possível na pagina dessa boa pratica discutir a mesma sob a forma de comentários
%   O site tem cerca de 70 boas praticas a maioria sugerida pelo 2 autores
%   Parece-me que a comunidade de votantes nas varias praticas é cerca de vinte pessoas
%   Mas muitas destas paginas de boas praticas já foram visitadas por mais de 5000 pessoas
%   Em parte estas pessoas devem advir da utilização criada pelo autor que valida automaticamente 
%   as praticas no aplicação e apresenta um relatório com link para a pagina onde se da a discussão.
%   Esta gem apenas tem implementadas 25 das 70 boas praticas.
%  
% Alem disto existe uma serie de gems para fazer diversas analises ao codigo, 
% e até uma que agrega estas. Existe um video aqui que mostra estas em funcionamento:
% http://media.railscasts.com/videos/252_metrics_metrics_metrics.mov
% 
% 2. Secção de questões:
%   O site tem ainda uma área de perguntas e respostas. Tipo forum de discução.
% 
% 
% Outro ponto interessante é que é mais fácil encontrar projectos ruby open source 
% (por exemplo o próprio rails, ou gems que tratam diversos problemas), do que applicações rails.
% Embora na realidade, provavelmente, existam mais aplicações rails do que projectos em ruby 
% (claro que estas se podem incluir nos projectos ruby) isto acontece porque as aplicações rails são o produto final, 
% fazem uso dos projectos ruby, mas na maior parte do caso não são reutilizadas.

%I conclude that the biggest part of ruby developers are in fact Ruby on Rails or Web developers, they tend to write HTML, javascript, CSS and use ruby gems maintained by small rubists communities (Ruby on rails is in fact a compilation of small projects). Rails developers write ruby code inside their rails app, but in many cases they have never wrote a ruby gem.
%There are a lot of frameworks and projects but I think that the only framework that deserves a big study is Ruby on Rails, all other frameworks have small communities compared to it. The other possibility is to find out best practices of ruby code in general, thinking of it as script code and not with a project structure.