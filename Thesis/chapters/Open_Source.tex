\thispagestyle{empty}
\chapter{Open Source}\label{chap:open_source}


%% Open Source Project Hosting Platforms}
\section{What is Open Source?}

Escrever, aqui!

os dois lados quem desenvolve e quem usa.

\section{Open Source Project Hosting Platforms}

\begin{table}[H]
\begin{threeparttable}
\begin{tabular}{|c|c|c|c|c|c|} \hline 
\multicolumn{6}{|c|}{Open Source Project Hosting Web Sites} \\ \hline 
Name & Established & Available VCS & Users & Projects & Alexa rank \tnote{a} \\\hline
\mr{5}{SourceForge}      &\mr{5}{1999}  &CVS            &\mr{5}{2,000,000}   &\mr{5}{236,319}          &\mr{5}{136}            \\
                         &              &SVN            &                    &                         &                       \\
                         &              &Bazar          &                    &                         &                       \\
                         &              &GIT            &                    &                         &                       \\
                         &              &Mercurial      &                    &                         &                       \\\hline 
\mr{1}{GitHub}           &\mr{1}{2008}  &GIT            &\mr{1}{505,000}     &\mr{1}{1,516,000}        &\mr{1}{742}            \\\hline 
\mr{2}{Google Code }     &\mr{2}{2006}  &SVN            &\mr{2}{?}           &\mr{2}{250,000}          &\mr{2}{900\tnote{b}}   \\
                         &              &Mercurial      &                    &                         &                       \\\hline 
\mr{3}{Code Plex}        &\mr{3}{2006}  &SVN            &\mr{3}{151,782}     &\mr{3}{15.955}           &\mr{3}{2,343}          \\
                         &              &Microsoft TFS  &                    &                         &                       \\
                         &              &Mercurial      &                    &                         &                       \\\hline 
\mr{2}{Assembla}         &\mr{2}{2006}  &SVN            &\mr{2}{180,000}     &\mr{2}{60,000}           &\mr{2}{6,628}          \\
                         &              &GIT            &                    &                         &                       \\\hline 
\mr{1}{Launchpad}        &\mr{1}{2005}  &Bazar          &\mr{1}{1,140,345}   &\mr{1}{19,016}           &\mr{1}{12,466}         \\\hline 
\mr{4}{BerliOS}          &\mr{4}{2000}  &CVS            &\mr{4}{47,285}      &\mr{4}{5,448}            &\mr{4}{17,299}         \\
                         &              &SVN            &                    &                         &                       \\
                         &              &GIT            &                    &                         &                       \\
                         &              &Mercurial      &                    &                         &                       \\\hline 
\mr{1}{Bitbucket}        &\mr{1}{2008}  &Mercurial      &\mr{1}{51,600}      &\mr{1}{27,769}           &\mr{1}{12,047}         \\\hline 
\mr{1}{Gitorious}        &\mr{1}{2008}  &GIT            &\mr{1}{?}           &\mr{1}{8,336}            &\mr{1}{28,531}         \\\hline 
\mr{5}{GNU Savannah}     &\mr{5}{2000}  &CVS            &\mr{5}{48,593}      &\mr{5}{3,233}            &\mr{5}{48,286}         \\
                         &              &SVN            &                    &                         &                       \\
                         &              &Bazar          &                    &                         &                       \\
                         &              &Arch           &                    &                         &                       \\
                         &              &GIT            &                    &                         &                       \\
                         &              &Mercurial      &                    &                         &                       \\\hline 
\end{tabular}
\begin{tablenotes}
  \item    Data retrieved on 2010-12-20, from each of the OSPH Websites, and using Alexa rank website.
  \item[a] Alexa rank represents the approximate number of websites, in the world, that have a higher popularity than the given site
           (the smaller the better).
  \item[b] This value is an approximation.
\end{tablenotes}
\end{threeparttable}
\caption{Open Source Project Hosting Websites}
\label{table:OSPHWebSites}
\end{table}

An open source project hosting platform is the central tool that supports and co-ordinates the development of an open source project,
 normally it is in a form of a website.

Since 1999 (year that SourceForge was launched), many OSPHW were created to host open source projects.
OSPH sites offer different features, 
like codebase \footnote{The term codebase means the whole collection of source code used to build a particular application or component.} 
hosting (a project codebase is typically stored in a source control repository), 
code review, bug tracking, web hosting, wiki, mailing list, etc~\cite{binkley2006animated}.

Table \ref{table:OSPHWebSites} shows a list of the most well known OSPHW.
By looking at this table we can see that SourceForge is the most well established OSPHW.
It is also one of oldest, 
hosts more than 230,000 projects and has more than 2 million registered users~\cite{christley2005collection}.

GitHub is one of the youngest OSPHW (launched in 2008). 
However, in only two years, it drew more than 500,000 users (one quarter of sourceforge users) and is hosting more than 1,500,000 projects. 
The only version control system provided by GitHub is 
git\footnote{\url{http://git-scm.com/}. 
  Git is a free \& open source, distributed version control system that Linus Torvalds developed to help manage Linux kernel development.
}.
Because GitHub projects are in fact git repositories, 
it is incredibly easy to make branches and merges in GitHub. 
Although branching was considered a bad practice, 
it turned out that using git, it can in fact improve the developers collaboration and organization.
Hosting a Git repository is not hard but co-ordinating efforts of forking and merging amongst people is tough. 
With a system like Github, it becomes a lot easier~\cite{petercooper2010}.
However, the main reason for GitHub popularity are the social aspects.
Users and projects have public profiles and activity feeds which display activity on public projects such as commits, comments, forks, etc.
Furthermore, with so many high profile projects on board (
jQuery, 
reddit, 
Sparkle, 
curl, 
Ruby on Rails, 
node.js, 
ClickToFlash, 
Erlang/OTP, 
CakePHP, 
Redis), 
it is easy to imagine that GitHub could be the next SourceForge.

The above are the reasons why I believe that to build a software, like the one described in the introduction, with focus on GitHub projects, would benefit a big open-source community. 

\section{Need to Assess Open Source Projects}

Escrever, aqui!

