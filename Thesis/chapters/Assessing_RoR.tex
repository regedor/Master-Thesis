\thispagestyle{empty}
\chapter{Assessing Ruby on Rails Projects}\label{chap:assissing_ror}

To assess the quality of a software project is not an easy task.
To many variables should be taken into consideration and most of their values can be considered subjective.

Our thesis is that an open source project quality is somehow influenced by the best practices followed 
by its contributors.

Furthermore, since Rails community has been our object of study we decided to conduct a series of studies, 
with the objective of finding correlations between best practices defined by the Rails community and
the quality of projects following it.

With the objective of automatically verify whether or not best practices are being followed by
a given Rails project, the open source ruby gem rails\_best\_practices 
was created (by the authors of Rails best practices web site).
We agreed on using it as starting point for our work.

This chapter reports the different studies carried out, the main difficulties and the results obtained.



\section{First Study}\label{subsec:first_study}
One of the first things that we have noticed when we applied this gem to OSS projects,
is that the biggest and most renown projects have much more errors than the smaller and unknown projects.
This nonsense has a simple  interpretation.
Small projects (like the majority of Rails projects found in GitHub) are simple software packages,
often developed by a single user, projects carried out for simple learning purposes.
These applications are so simple that many times the code is almost entirely created by Rails code generators.
Usually, when code is not written by humans, it has few mistakes concerning those recommendations.

Having taken the above into account, we decided to run the rails best practices gem on similar Rails projects.
Seven \emph{time tracking} or \emph{project management} open source systems were chosen.
After running the gem and counting the
\textsf{not best practices (NBPs)}\footnote{In fact, Rails best practices gem does not find best practices in the source code.
  It does the opposite, it discovers when the code is not written according to a best practice, in other words, 
  it identifies bad practices (similar to the detection of code smells).
  We decided to name those occurrences NBP.
}
occurrences, the following results were obtained:

\begin{table}[H]
\begin{center}{\scriptsize
  \begin{threeparttable}
  \begin{tabular}{|l||c|c|c|c|c|c|c|} \hline
  \multicolumn{8}{|c|}{Rails Best Practices Results} \\ \hline
  \textbf{Best Practice}& \textbf{A}& \textbf{B}& \textbf{C}&  \textbf{D}& \textbf{F}& \textbf{G}& \textbf{H} \\\hline\hline
  \emph{\tnote{a}Add model virtual attribute           }              &   -  &   2  &   7  &   - &   - &   5 &   4  \\ \hline
  \emph{Always add db index                   }              &   -  &   -  &   -  &  43 &   - &   - &  51  \\ \hline
  \emph{Isolate seed data                     }              &   -  &   -  &   -  &   - &   - &  79 &  17  \\ \hline
  \emph{Law of demeter                        }              &  20  &  38  &  45  &   6 &  30 & 164 &  85  \\ \hline
  \emph{Move code into controller             }              &   -  &   -  &   -  &   - &   2 &   - &   4  \\ \hline
  \emph{Move code into model                  }              &   -  &  26  &   -  &   7 &   1 &   3 &  19  \\ \hline
  \emph{Move model logic into model           }              &   -  &   -  &  76  &  11 &  11 &  98 & 100  \\ \hline
  \emph{Move finder to named\_scope           }              &   -  &   4  &   9  &   2 &   4 &  25 &   -  \\ \hline
  \emph{Needless deep nesting                 }              &   -  &   -  &   -  &   1 &   - &   - &   -  \\ \hline
  \emph{Not use default root                  }              &   -  &   1  &   1  &   - &   1 &   1 &   1  \\ \hline
  \emph{Notes  use query attribute            }              &   -  &   2  &   -  &   - &   - &   - &   -  \\ \hline
  \emph{Overuse route customizations          }              &   -  &   -  &   2  &   4 &   - &   2 &   2  \\ \hline
  \emph{Remove trailing whitespace            }              &  68  &  57  & 126  & 110 & 330 & 316 & 100  \\ \hline
  \emph{Use factory method                    }              &   -  &  15  &   9  &   5 &   1 &   8 &  19  \\ \hline
  \emph{Replace instance var with local var   }              &  13  &   -  &  70  & 239 & 142 &  31 & 100  \\ \hline
  \emph{Use before\_filter                    }              &   -  &   7  &   9  &   8 &   8 &  19 &  23  \\ \hline
  \emph{Wrong email content\_type             }              &   -  &   3  &   -  &   - &   - &   - &   -  \\ \hline
  \emph{Use query attribute                   }              &   -  &   -  &  11  &   5 &   8 &  29 &   6  \\ \hline
  \emph{Use say with time in migrations       }              &   -  &   -  &  24  &   - &  10 &  23 &  56  \\ \hline
  \emph{Use scopes access                     }              &   -  &   -  &   -  &   - &   - &   - &  04  \\ \hline
  \emph{User model association                }              &   -  &   -  &  12  &   9 &   - &   1 &  21  \\ \hline
  \emph{Keep finders on their own model       }              &   8  &   4  &   1  &   - &  11 &   - &   -  \\ \hline
  \emph{Total                                 }              & 109  & 156  & 402  & 450 & 559 & 834 & 864  \\ \hline
  \end{tabular}
  \begin{tablenotes}
	\item Results obtained by running the \emph{best practices analyzer gem} on the 7 Open Source Projects chosen 
	      (data produced on April, 2011).
    \item \emph{A:} Rubytime
    , \emph{B:} Notes
    , \emph{C:} Tracks
    , \emph{D:} Handy Ant
    , \emph{F:} Retrospectiva
    , \emph{G:} Redmine
    , \emph{H:} Clockingit
    \item Figures shown represent the number of times a project do not follow a best practice; is expected that \emph{smaller the number, better the project}.
  \end{tablenotes}
  \end{threeparttable}
}
\end{center}
\caption{Best practices analyzer gem raw results using 7 open source projects}
\end{table}

Rubytime seems to have the best results and Clockingit the worst. 
The fact is that very good user reviews can be found about Rubytime.
However, Tracks obtained an unexpected high score, since it has been very sparsely maintained 
(old code has higher probability of not following the current best practices).
As explained before, those values are not really measuring if a project follows best practices 
but instead measuring when it fails.
This should also be taken into consideration. 

The most evident problem here is that best practices are not being weighted and neither the size of the project considered.
For instance, if the developers have the habit of leaving trailing white spaces, 
the occurrences of this will obviously be related to the size of the project.
On the other hand, it is a best practice to remove the default route generated by rails, 
independently of the project size this is true or false, there is no way to leave the route two times. 
So, if developers do not take into account those two best practices, when the project grows, 
the number of trailing spaces will increase and the results will show more NBPs, 
but the other one will always be only one NBP.  
Because of that we can get twisted results.

To avoid this, the projects were sized.
The size attribute is based on the quantity of models and controllers in the project.
After that, we divided the values previously obtained  by the project size.
By doing that, a new set of results emerge.

\begin{table}[H]
\begin{center}
{\scriptsize
\begin{threeparttable}
\begin{tabular}{|l||c|c|c|c|c|c|c|} \hline
\multicolumn{8}{|c|}{Rails Best Practices Results} \\ \hline
\textbf{Best Practice}& \textbf{A}& \textbf{B}& \textbf{C}&  \textbf{D}& \textbf{F}& \textbf{G}& \textbf{H} \\\hline\hline
\emph{Total                                           }              & 109  & 156  & 402  & 450 & 559 & 834 & 864  \\ \hline
\emph{Total Without Trailing Whitespace               }              &  41  &  99  & 276  & 340 & 229 & 518 & 764  \\ \hline
\emph{Project Size                                    }              &  12  &  11  &  11  &  29 &  26 &  58 &  31  \\ \hline
\emph{Total / Project Size                            }              &   9  &  14  &  37  &  16 &  23 &  15 &  28  \\ \hline
\emph{Total Without Trailing Whitespace / Project Size}              &   3  &   9  &  25  &  12 &   9 &   9 &  25  \\ \hline
\end{tabular}
\begin{tablenotes}
  \item \emph{A:} Rubytime
  ; \emph{B:} Notes
  ; \emph{C:} Tracks
  ; \emph{D:} Handy Ant
  ; \emph{F:} Retrospectiva
  ; \emph{G:} Redmine
  ; \emph{H:} Clockingit
  \item Results obtained by running the \emph{best practices analyzer gem} on the 7 Open Source Projects chosen, 
        after normalization (data produced on April, 2011).
\end{tablenotes}
\end{threeparttable}
}
\end{center}
\caption{Best practices analyzer gem normalized results using 7 open source projects}
\label{tab:rbpresults_1}
\end{table}

Those results are much more likely to be helpful in terms of understanding if a project is or is not following 
best practices.
The numbers reflect both the community reviews and our own estimates much more.


\section{Second Study}\label{subsec:second_study}
After the first study reported above, we felt that it was time to make a bigger one;
we should repeat the experiment over a larger sample. 
In addition, there was the need to define an objective quality metric to compare the metrics results with.
As a second target for this new phase, it was decide to find an objective quality rate (a reputation ranking) for each project in the sample, 
to be possible to compare with the results computed for the best practices metrics.

For the second study, we selected 40 Ruby on Rails projects hosted in github and
decided to consider the number of 
\textsf{followers}\footnote{Number of users that want to receive notifications about the project.} and
\textsf{forks}\footnote{Number of people that forked the project. This means that either they want to contribute to the project or create a derived project}, 
that each project has on github, 
as a \emph{project reputation} metric. 

The objective was to prove that a negative correlation exists, between the NBPs of a project and its followers and forks. 

The previous study has shown us the need to apply different weights to each NBP. 
By diving the NBPs by the project size, in the first study, seemed like we got better results.
However, not all NBPs depend on the project size. 
Therefore, we altered the rails best practices gem to make it possible to know how much project files were analyzed 
by each rails best practice checker.

Basically, after collecting the GitHub URLs for each project, we followed the next steps:
\begin{itemize}
\item \emph{Retrieve GitHub information}, in this step we get the followers and forks(and more info that might be used in further analyses).
\item \emph{Download the project repository}.
\item \emph{Run rails best practices gems}, at this point, we get the non weighted NBPs and files given by each one of the 29 checkers.
\item \emph{Calculate the Weighted Global NBPs}, the evaluation algorithm consists in dividing the value returned by each NBP checker  by the number of files checked and, then sum it.
\end{itemize}

Next, an excerpt of the obtained table is shown:
\begin{table}[H]
\begin{center}
{\scriptsize
\begin{threeparttable}
\begin{tabular}{|l||c|c|c|c|c|c|c|c|c|c|c|} \hline
\multicolumn{12}{|c|}{Rails Best Practices Results} \\ \hline
Projects & \textbf{Forks}         & \textbf{Watchers} & 
C1       & C1 F.                  & \textbf{W. C1} & 
C2       & C2 F.                  & \textbf{W.       C1} & 
...      & T. NBPs                & \textbf{W.  T. NBPs} \\\hline\hline
\emph{Rails Admin } & 30 & 2478 &  0 & 141 & \textbf{  0 }&  0 &  37 & \textbf{ 0} & ...&  50 & \textbf{ 739}  \\ \hline
\emph{Rubytime    } & 12 &   82 & 24 & 161 & \textbf{149 }&  0 & 134 & \textbf{ 0} & ...& 146 & \textbf{1334}  \\ \hline
\emph{Redmine     } & 30 & 1781 & 49 & 996 & \textbf{ 49 }&  1 & 362 & \textbf{ 2} & ...& 884 & \textbf{1402}  \\ \hline
\emph{BrowserCMS  } & 30 &  784 & 11 & 234 & \textbf{ 47 }&  0 & 216 & \textbf{ 0} & ...& 268 & \textbf{1510}  \\ \hline
\emph{Tracks      } & 17 &   87 & 46 & 842 & \textbf{ 54 }& 15 & 271 & \textbf{55} & ...& 569 & \textbf{2810}  \\ \hline
\emph{...}&...&...&...&...&...&...&...&...&...&...&...\\ \hline
\end{tabular} 

\begin{tablenotes}
  \item Results obtained by running the \emph{best practices analyzer gem} on the 40 Open Source Projects chosen, 
        from GitHub (data produced on April, 2011). The full table can be found at www.study.gorgeouscode.com
  \item{ \emph{C(x): }} The rails best practices gem has 29 checkers(when this study was carried), 
                        each one tries to find occurrences of a different nbp in the project. 
  \item{\emph{C(x) Files: }} The number of files in the project, where it tried to find nbps 
                             (for instance, some checkers may only be concerned with html files, 
                             some other checker nbps my only occur in model files, etc)
  \item{\emph{W. C(x): }} Weighted C(x) = C(X) / C(x)Files * 1000 
                          (A really small number is added to each variable to avoid divisions by zero).
\end{tablenotes}
\end{threeparttable}
}
\end{center}
\caption{Best practices analyzer gem results using 40 open source projects}
\end{table}

\section{Results}\label{subsec:results}
After building the table containing the results for the 40 projects, we easily found correlations between columns.
We discovered that the average correlation index, for the weighted C(x) columns, is -0.2. Only three of the weighted C(x) columns do not have negative correlation. This is quite good, considering the fact that there is an explanation for it. 
Those three checkers (without negative correlation) aimed at finding  NBPs that almost non of the projects were committing, 
so there is no correlation.


The most important results are in the next table:
\begin{table}[H]
\begin{center}
{\scriptsize
\begin{threeparttable}
\begin{tabular}{|l||c|c|} \hline
\multicolumn{3}{|c|}{Correlations} \\ \hline
                       & \textbf{Total NBPs}  & \textbf{Total Weighted NBPs}  \\ \hline\hline
\emph{Forks         }  & 0.14                 & -0.53                       \\ \hline
\emph{Watchers      }  & 0.07                 & -0.40                       \\ \hline
\end{tabular}
\begin{tablenotes}
  \item www.study.gorgeouscode.com for the complete table.
\end{tablenotes}
\end{threeparttable}
}
\end{center}
\caption{Relations between NPBs forks and watchers}
\end{table}

These correlation indexes show that if we just count the nbps there is no relation between them and the number of forks and watchers. Nevertheless, the Weighted NBPs have a quite perceptible negative correlation both with watchers and forks. 

Observing that Table, it is possible to notice that the forks correlation is bigger. 
We believe that if it happens, it is because forking a project shows intensions of digging into the code and, 
of course, it easier to understand others code when it follows good practices.




\section{Third Study}\label{subsec:thrid_study}
After the previous studied, it turned out that we have a achieved a new metric for classifying the quality of Rails projects.
By inverting the total number of NBPs and scaling it, we can create a tool to automatically give a score to the projects.

That is exactly what we did, we created a simple application that receives the path for a GitHub rails project. 
As result by automating everything, it was possible to run the application against a bigger set of projects.
A list of analyzed projects can be found at \url{www.study.gourgeouscode.com}.





\subsection{Best Practices Considered}\label{subsec:second_study}


\subsubsection{Remove Tab}
  Make sure there are no tabs in files.
 
  % See the best practice details here \url{http://rails-bestpractices.com/posts/81-remove-tab}
   
\subsubsection{Remove Trailing Whitespace }  
  Make sure there are no trailing whitespace in codes.
 
  % See the best practice details here \url{http://rails-bestpractices.com/posts/60-remove-trailing-whitespace}
   
\subsubsection{Add Model Virtual Attribute}
  Make sure to add a model virual attribute to simplify model creation.
 
  % See the best practice details here \url{http://rails-bestpractices.com/posts/4-add-model-virtual-attribute}
       
\subsubsection{Always Add Db Index}
  Review db/schema.rb file to make sure every reference key has a database index.
 
  % See the best practice details here \url{http://rails-bestpractices.com/posts/21-always-add-db-index}
       
\subsubsection{Dry Bundler In Capistrano}
  Review config/deploy.rb file to make sure using the bundler's capistrano recipe.
 
  % See the best practice details here \url{http://rails-bestpractices.com/posts/51-dry-bundler-in-capistrano}
   
\subsubsection{Isolate Seed Data }
  Make sure not to insert data in migration, move them to seed file.
 
  % See the best practice details here \url{http://rails-bestpractices.com/posts/20-isolating-seed-data.}
   
\subsubsection{Keep Finders On Their Own Model}
  Review model files to ake sure finders are on their own model.
 
  % See the best practice details here \url{http://rails-bestpractices.com/posts/13-keep-finders-on-their-own-model.}
   
\subsubsection{Law Of Demeter }
  Review to make sure not to avoid the law of demeter.
 
  % See the best practice details here \url{http://rails-bestpractices.com/posts/15-the-law-of-demeter.}
   
\subsubsection{Move Code Into Controller}
  Review a view file to make sure there is no finder, finder should be moved to controller.
 
  % See the best practice details here \url{http://rails-bestpractices.com/posts/24-move-code-into-controller.}
   
\subsubsection{Move Code Into Helper }
  Review a view file to make sure there is no complex options\_for\_select message call.
 
  % See the best practice details here \url{http://rails-bestpractices.com/posts/26-move-code-into-helper.}
   
\subsubsection{Move Code Into Model }
  Review a view file to make sure there is no complex logic call for model.
 
  % See the best practice details here \url{http://rails-bestpractices.com/posts/25-move-code-into-model.}
     
\subsubsection{Move Finder To Named Scope }
  Review a controller file to make sure there are no complex finder.
 
  % See the best practice details here \url{http://rails-bestpractices.com/posts/1-move-finder-to-named_scope.}
   
\subsubsection{Move Model Logic Into Model}
  Review a controller file to make sure that complex model logic should not exist in controller, should be moved into a model.
 
  % See the best practice details here \url{http://rails-bestpractices.com/posts/7-move-model-logic-into-the-model.}
   
\subsubsection{Needless Deep Nesting}
  Review config/routes.rb file to make sure not to use too deep nesting routes.
 
  % See the best practice details here \url{http://rails-bestpractices.com/posts/11-needless-deep-nesting.}
   
\subsubsection{Not Use Default Route}
  Review config/routes file to make sure not use default route that rails generated.
 
  % See the best practice details here \url{http://rails-bestpractices.com/posts/12-not-use-default-route-if-you-use-restful-design}
   
\subsubsection{Not Use Time Ago In Words}
  Review view and helper files to make sure not use time\_ago\_in\_words or distance\_of\_time\_in\_words\_to\_now.
 
  % See the best practice details here \url{http://rails-bestpractices.com/posts/105-not-use-time_ago_in_words.}
   
\subsubsection{Overuse Route Customizations}
  Review config/routes.rb file to make sure there are no overuse route customizations.
 
  % See the best practice details here \url{http://rails-bestpractices.com/posts/10-overuse-route-customizations.}
   
\subsubsection{Protect Mass Assignment}
  Review model files to make sure to use attr\_accessible or attr\_protected to protect mass assignment.
 
  See the best practices details here \url{http://rails-bestpractices.com/posts/148-protect-mass-assignment.}
   
\subsubsection{Remove Empty Helpers Review}
  Review a helper file to make sure it is not an empty moduel.
 
  % See the best practice details here \url{http://rails-bestpractices.com/posts/72-remove-empty-helpers.}
   
\subsubsection{Remove Unused Methods In Controllers}
  Find out unused methods in controllers.
   
\subsubsection{Remove Unused Methods In Helpers}
  Find out unused methods in helpers.
       
\subsubsection{Remove Unused Methods In Models}
  Find out unused methods in models.
   
\subsubsection{Replace Complex Creation With Factory Method}
  Review a controller file to make sure that complex model creation should not exist in controller, should be replaced with factory method.
 
  % See the best practice details here \url{http://rails-bestpractices.com/posts/6-replace-complex-creation-with-factory-method.}
   
\subsubsection{Replace Instance Variable With Local Variable}
  Review a partail view file to make sure there is no instance variable.
 
  % See the best practice details here \url{http://rails-bestpractices.com/posts/27-replace-instance-variable-with-local-variable.}
   
\subsubsection{Restrict Auto Generated Routes}
  Review a route file to make sure all auto-generated routes have corresponding actions in controller.
 
  % See the best practice details here \url{http://rails-bestpractices.com/posts/86-restrict-auto-generated-routes}
   
\subsubsection{Simplify Render In Controllers}
  Review a controller file to make sure using simplified syntax for render.
 
  % See the best practice details here \url{http://rails-bestpractices.com/posts/62-simplify-render-in-controllers.}
   
\subsubsection{Simplify Render In Views}
  Review a view file to make sure using simplified syntax for render.
 
  % See the best practice details here \url{http://rails-bestpractices.com/posts/61-simplify-render-in-views.}
   
\subsubsection{Use Before Filter}
  Review a controller file to make sure to use before\_filter to remove duplicated first code line in different action.
 
  % See the best practice detailed here \url{http://rails-bestpractices.com/posts/22-use-before_filter.}
   
\subsubsection{Use Model Association}
  review a controller file to make sure to use model association instead of foreign key id assignment.
 
  % See the best practice details here \url{http://rails-bestpractices.com/posts/2-use-model-association.}
   
\subsubsection{Use Multipart Alternative As Content Type Of Email}
  Make sure to use multipart/alternative as content\_type of email.
 
  % See the best practice details here \url{http://rails-bestpractices.com/posts/41-use-multipart-alternative-as-content\_type-of-email.}
   
\subsubsection{Use Observer}
  Make sure to use observer (sorry we only check the mailer deliver now).
 
  % See the best practice details here \url{http://rails-bestpractices.com/posts/19-use-observer.}
   
\subsubsection{Use Query Attribute}
  Make sure to use query attribute instead of nil?, blank? and present?.
 
  % See the best practice details here \url{http://rails-bestpractices.com/posts/56-use-query-attribute.}
   
\subsubsection{Use Say With Time In Migrations}
  Review a migration file to make sure to use say or say\_with\_time for customized data changes to produce a more readable output.
 
  % See the best practice details here \url{http://rails-bestpractices.com/posts/46-use-say-and-say\_with\_time-in-migrations-to-make-a-useful-migration-log.}
   
\subsubsection{Use Scope Access}
  Review a controller to make sure to use scope access instead of manually checking current\_user and redirect.
 
  % See the best practice details here \url{http://rails-bestpractices.com/posts/3-use-scope-access.}




















