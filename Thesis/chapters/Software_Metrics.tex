\thispagestyle{empty}
\chapter{Software Metrics}\label{chap:software_metrics}

In this chapter, you can learn the basics of software metrics. 
It is given a brief explanation of quality attributes. Furthermore, we stand that, in the context of open source software, maintainability is the most relevant attribute. In addition, a few traditional software metrics are listed.


%%--------------------------------------
%% Assessing Open Source Software
\section{Assessing Open Source Software} \label{sec:assessing}
The simplest operation in science and the lowest level of measurement is classification~\cite{kan2002metrics}.

By assessing OSS we mean to sort OSS projects into an
\textsf{ordinal scale}\footnote{Ordinal scale refers to the measurement operations through which the subjects can be compared in order.}
This can be achieved by defining a
\textsf{ranking system}\footnote{
  Raking system example: to classify a quality attribute, for instance the project documentation, according to its quality with
  five, four, three, two or one star.
} and by placing OSS projects into quality categories with respect to certain quality attributes.
First we need to find a way of quantifying those OSS quality attributes.

%% Sofware Quality
In software, quality is an abstract concept. It is commonly recognized as lack of "bugs", and the meeting of the functional requirements.
However, quality can be perceived and interpreted differently based on the actual context, objectives and interests of each project.
Many software development companies do monitor costumer satisfaction as a quality index,
for instance, IBM ranks their software products in levels of CUPRIMDSO~\cite{kan2002metrics}:
\begin{itemize}
\item Capability/Functionality (refers to the software meeting its functional requirements)
\item Usability (refers to the required effort to learn, and operate the software)
\item Performance/Efficiency (refers to the software performance and resource consumption)
\item Reliability (refers to software fault tolerance and recoverability)
\item Instalability/Portability (refers to the required effort to install or transfer the software to another environment)
\item Maintainability (refers to the required effort to modify the software)
\item Documentation/Information (refers to the coverage and accessibility of the software documentation)
\item Service (refers to the company monitoring and service)
\item Overall (refers to an overall classification based on the other attributes)
\end{itemize}
Almost every big software company have similar quality attributes.
ISO/IEC 9126 provides a framework for the evaluation of software quality (The goal is to achieve quality in use, in other words, quality from the user perspective)~\cite{bevan1999quality}
IISO/IEC 912 defines six software quality attributes:
\begin{itemize}
\item Functionality (refers to the software meeting of the functional requirements)
\item Reliability (refers to software fault tolerance and recoverability)
\item Usability (refers to the required effort to learn, and operate the software)
\item Efficiency (refers to the software performance and resource consumption)
\item Maintainability (refers to the required effort to modify the software)
\item Portability (refers to the required effort to transfer the software to another environment)
\end{itemize}
Quality attributes have interrelationships.
They can be
conflictive\footnote{Conflictive, negative influence, if one attribute is high it makes the other one low.} or
support\footnote{Support, positive influence, if one attribute is high it makes the other one high too.} one another.
For example, the higher the functional complexity of the software, the harder it becomes to achieve maintainability~\cite{kan2002metrics}.

Because of the OSP bazaar style and continuous development process, it is intuitive that the maintainability and documentation attributes
have a big influence on the overall quality and continuous progress of an OSP
(maintainability and documentation have support relationships
with usability, reliability and availability attributes, but might be conflictive with functionality and performance attributes).

Failure to meet functionality often leads to late changes and increased costs in the development process.
The software industry and researchers have been mostly interested on testing methodologies
that focus on functional requirements and pay little attention to non-functional requirements~\cite{chung2009non}.

There are several challenges and difficulties, in assessing non-functional quality attributes for software projects.
For example, security is a non-functional requirement that needs to be addressed in every software project.
Therefore a badly-written software may be functional, but subject to buffer overflow attacks.
Another example is the amount of codebase comments, if the code does not have any comments it will not affect the functional requirements,
but it is obvious that it will decrease readability and maintainability~\cite{gousios2007software}.

%% SOFTWARE METRICS
\subsection{Classic Software Metrics}
To classify OSS with regards to a certain quality attribute, we need to find which factors influence it.
Then we need a way to measure that attribute. If we need to measure we need metrics.

Fortunately, there are around two thousand documented software metrics, but there is few information on how those metrics relate to each other.
Most of them simply have different names but give similar information~\cite{fenton1999software}.
The major challenge is to discover how important the information given by those metrics is, if the calculation effort pays off,
how to interpret their values and find
\textsf{correlations}\footnote{Correlation is probably the most widely used statistical method to assess relationships among observational data~\cite{kan2002metrics}.}
 to assess the quality attributes of an OSP.


\subsubsection{Lines of Code}
A line of code is any line of program text that is not a comment or blank line, 
regardless of the number of statements or fragments of statements in the line. 
This specifically includes all lines containing program headers, declarations, and executable and non-executable statements.~\cite{conte1986software}
The lines of code (LOC) metric is anything but simple. 
The major problem comes from the ambiguity of the operational definition, the actual counting.
In the early days of Assembler programming, in which one physical line was the same as one instruction, the LOC definition was clear.
With the availability of high-level languages the one-to-one correspondence broke down.
Differences between physical lines and instruction statements (or logical lines of code) 
and differences among languages contribute to the huge variations in counting LOC.
For instance, Boehm~\cite{boehm2009software} counts lines as physical lines and includes executable lines, data definitions, and comments. 
Even within the same language, the methods and algorithms used by different counting tools can cause significant differences in the final counts.

Next, the most common variations are described~\cite{jones1986programming}: 
\begin{itemize}
\item Count only executable lines. 
\item Count executable lines plus data definitions. 
\item Count executable lines, data definitions, and comments. 
\item Count executable lines, data definitions, comments, and job control language. 
\item Count lines as physical lines on an input screen. 
\item Count lines as terminated by logical delimiters. 
\end{itemize}


\subsubsection{Cyclomatic Complexity}
The measurement of cyclomatic complexity~\cite{mccabe1976complexity} was designed to indicate a program's testability and maintainability.
It is the classical graph theory cyclomatic number, indicating the number of regions in a graph.
As applied to software, it directly measures the number of linearly independent paths through a program source code.
Basically it counts how many different executions a program can have. 
For instance, one "if" statement will double the number of paths. 
Therefore, a high number of control statements (ifs, loops, etc) in a program source code will result in a high cyclomatic complexity. 
As such it can be used to indicate the effort required to test a program. To determine the paths, the program procedure is represented as a strongly connected graph with unique entry and exit points. The general formula to compute cyclomatic complexity is: 
$$M = V(g) = e - n + 2p$$
where
\begin{itemize}
\item \emph{$V(g)$} is the cyclomatic number of g 
\item \emph{$e$   } is the number of edges 
\item \emph{$n$   } is the number of nodes 
\item \emph{$p$   } is the number of unconnected parts of the graph 
\end{itemize}


\subsubsection{Fan-In and Fan-Out}
Fan-in and fan-out are perhaps the most common design structure metrics, which are based on the ideas of coupling~\cite{yourdon1979structured}:
\begin{itemize}
\item \emph{Fan-in } is a count of the modules that call a given module
\item \emph{Fan-out} is a count of modules that are called by a given module
\end{itemize}

In general, modules with a large fan-in are relatively small and simple, and are usually located at the lower layers of the design structure.
In contrast, modules that are large and complex are likely to have a small fan-in.
There is also the theory that high fan-outs represent a high number of method calls and thus are undesirable,
while high fan-ins represent a high level of reuse~\cite{wang2007dynamic}.

\subsubsection{Object-Oriented Metrics}
Classes and methods are the basic constructs of OO technology.
The amount of function provided by an OO software can be estimated based on the number of identified classes and methods or their variants.
Therefore, it is natural that the basic OO metrics are related to classes, methods and their size.

The pertinent question therefore is what should the optimum value be for OO metrics. There may not be one correct answer,
but based on his experience in OO software development, Lorenz proposed eleven metrics as OO design metrics called rules of thumb~\cite{lorenz1994object}.
\begin{itemize}
\item \emph{Average Method Size (LOC)}: Should be less than 8 LOC for Smalltalk and 24 LOC for C++
\item \emph{Average Number of Methods per Class}: Should be less than 20. Bigger averages indicate too much responsibility in too few classes.
\item \emph{Average Number of Instance Variables per Class}: Should be less than 6. More instance variables indicate that one class is doing more than it should.
\item \emph{Class Hierarchy Nesting Level (Depth of Inheritance Tree, DIT)}: Should be less than 6, starting from the framework classes or the root class.
\item \emph{Number of Class/Class Relationships in Each Subsystem}: Should be relatively high. This item relates to high cohesion of classes in the same subsystem. If one or more classes in a subsystem don't interact with many of the other classes, they might be better placed in another subsystem.
\item \emph{Average Number of Comment Lines (per Method)}: Should be greater than 1.
\end{itemize}




%%%%%%%%%%%%%%%%%%%%%%%%%%%%%%%%%%%%%%%%%%%%
%% AVAILABLE TOOLS
\subsection{Available Tools}
Tools are intended to make a task easier.
Codebase analysis tools are designed turn the task of analyse and measure source code easier,
Those tools help in the process of finding software flaws and can also serve as aids for assessing the quality of software.

Next a list of some free tools:
\begin{itemize}
\item\textsf{FindBugs} find "bugs" in Java Programs
\item\textsf{FxCop (Microsoft)} analyses managed code assemblies and reports information about the assemblies
\item\textsf{PMD} scans Java source code and looks for potential code problems
\item\textsf{PreFast} (Microsoft) is a static analysis tool that identifies defects in C/C++ programs
\item\textsf{RATS} (Fortify) scans C, C++, Perl, PHP and Python source code for security problems like buffer overflows and Time Of Check, Time Of Use race conditions
\item\textsf{SWAAT} simplistic tool for Java, JSP, ASP .Net, and PHP
\item\textsf{Flawfinder} scans C and C++
\item\textsf{Saikuro} is a Ruby cyclomatic complexity analyzer.
\item\textsf{Reek} is a tool that examines Ruby classes, modules, methods and reports any code smells it finds.
\item\textsf{Rails best practices} is another tool that examines Ruby classes, modules, methods and reports any code smells it finds.




\end{itemize}

Tools like the ones listed above analyze code without executing it and point out what they consider to be potential weaknesses.
The most typical example of what those tools can find probably is calls to the emph{gets} function in
the C programming language: this function is inherently insecure and can lead to buffer
overflows. Specially crafted user input values can, for instance, allow an attacker to access
or modify confidential data or even take control of any computer executing that piece of
software.

Because these tools need to "understand" the code being analyzed, they are necessarily very language specific.
Furthermore, when analyzing a large amount of software projects, for instance Github projects,
it is of great importance to have a previous knowledge of what kind of projects  are going to be found (programing languages, frameworks, and other attributes),
this way tools can be adapted to get better results. 

