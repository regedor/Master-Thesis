\thispagestyle{plain}
\chapter*{Resumo}\label{chap:resumo}


{\it
  Este documento apresenta uma tese de mestrado em Ciência da Computação, 
  na área de \textit{compreensão do programas e análise estática de código}.

  Este trabalho integra o plano curricular do segundo ano de mestrado,
  e será defendido na Universidade do Minho em Braga, Portugal.

  São milhares os projetos de software open source disponíveis para colaboração em plataformas como o Github ou Sourceforge.
  No entanto, tal como o software tradicional, os projetos OOS diferem entre si a nivel de qualidade.
  Neste sentido, surge a necessidade do programador ou o utilizador final, saber o grau qualidade de um determinado projecto,
  antes mesmo de decidir iniciar uma colaboração ou o seu uso. 
  Sendo que para este efeito deve existir confiança no pacote de software, antes de qualquer tipo de decisão sobre a matéria.
   
  No contexto de software open source a confiabilidade surge como uma preocupação relevante, 
  sendo que, muitas das vezes os utilizadores finais preferem adquirir 
  software proprietário com o intuito de sentirem níveis de confiança mais elevados em relação ao referido pacote de software.
  Mais se refere que a avaliação de projetos de software open source pode ser executada de forma idêntica à de 
  software tradicional, utilizando métricas de software de conhecimento geral.
  
  Neste documento, pretendemos ultrapassar este ponto e propor um novo processo para fazer a dita análise de qualidade,
  ajustado precisamente para este ambiente de desenvolvimento de software único.
  Como é sabido, ao longo dos últimos anos, as comunidades de código aberto têm criado suas próprias regras e \emph{boas práticas}.
  No entanto, as métricas de software clássicas não têm em conta as \emph{boas práticas}
  estabelecidas pela comunidade.
  Nós sentimos que poderia valer a pena considerar essa peculiaridade como uma fonte complementar de dados de avaliação.
  
  Tomando como base de trabalho a comunidade open source do framework Ruby on Rails,
  o presente artigo discute o papel das
  \emph{boas práticas} na medição da qualidade de software, descreve os estudos realizados e
  a construção de um novo código de ferramenta de análises que
  permitirá aos usuários optar por melhores soluções, nomeadamente no que se refere ao tipo de software a utilizar e ainda 
  ajudar os developers a melhorar a qualidade do seu software.
}
