\thispagestyle{plain}
\chapter*{Resumo}\label{chap:resumo}


{\it
  Este documento apresenta uma tese de mestrado em Ciência da Computação, 
  na área de \textit{compreensão do programas e análise estática de código}.

  Este trabalho (preparação de tese e escrita) é um componente do segundo ano do mestrado,
  e será defendido na Universidade do Minho em Braga, Portugal.

  Milhares de projectos de software open source estão disponíveis para colaboração 
  em plataformas como o Github ou Sourceforge.
  No entanto, como o software tradicional, projetos OOS ter diferentes níveis de qualidade.
  O programador, ou o utilizador final, precisama saber a qualidade de um determinado projeto antes de iniciar a colaboração
  ou seu uso --- eles precisam, naturalmente, de confiar no pacote antes de tomar uma decisão.
  
  No contexto de software open source, confiabilidade é uma preocupação muito  sensível; 
  muitas das vezes os utilizadore finais preferem geralmente pagar
  software proprietário, para se sentirem mais confiantes na qualidade do pacote.
  Nada impede a avaliação de Projetos de software open source de da mesma forma que se faz a de 
  software tradicional usando as métricas de software bem conhecidos.
  
  Neste documento, queremos ir mais longe e propor um processo de grão mais fino para fazer a dita análise de qualidade,
  ajustado precisamente para este ambiente de desenvolvimento único.
  Como é sabido, ao longo dos últimos anos, as comunidades de código aberto têm criado suas próprias regras e \emph{boas práticas}.
  No entanto, as métricas de software clássicas não levam em conta as \emph{boas práticas}
  estabelecidas pela comunidade.
  Nós sentimos que poderia valer a pena considerar essa peculiaridade como uma fonte complementar de dados de avaliação.
  
  Tomando como base de trabalho a comunidade open source do framework Ruby on Rails,
  o presente artigo discute o papel das
  \emph{boas práticas} na medição da qualidade de software, descreve os estudos realizados, 
  a construção de um novo código de ferramenta de análises que
  permitirá aos usuários fazer melhores escolhas, sobre o que o software de usar e ajudar os desenvolvedores, 
  para melhorar o seu software.
}
