\thispagestyle{plain}
\chapter*{Resumo}\label{chap:resumo}


{\it
  Este documento apresenta uma tese de mestrado em Ciência da Computação, 
  na área de \textit{compreensão do programas e análise estática de código}.

  Este trabalho (preparação de tese e escrita) é um componente do segundo ano do mestrado,
  e será realizado na Universidade do Minho em Braga, Portugal.

  Milhares de projectos de software de open source (OOS) estão disponíveis para a colaboração 
  em plataformas como o Github ou Sourceforge.
  No entanto, como o software tradicional, projetos OOS ter diferentes níveis de qualidade.
  O desenvolvedor, ou o usuário final, precisamos saber a qualidade de um determinado projeto antes de iniciar a colaboração
  ou seu uso --- eles podem, naturalmente, para confiar no pacote antes de tomar uma decisão.
  
  No contexto da OSS, Confiabilidade é uma preocupação muito mais sensível; principalmente usuários finais geralmente preferem pagar
  software proprietário, para se sentir mais confiantes na qualidade do pacote.
  Projetos de OSS pode ser avaliada como pacotes de software tradicionais usando as métricas de software bem conhecidos.
  
  Neste artigo, queremos ir mais longe e propor um processo de grão mais fino para fazer análise de qualidade tal,
  ajustados precisamente para este ambiente de desenvolvimento único.
  Como é sabido, ao longo dos últimos anos, as comunidades de código aberto têm criado suas próprias regras e \emph{boas práticas}.
  No entanto, as métricas de software clássicos não levam em conta as \emph{boas práticas}
  estabelecido pela comunidade.
  Nós sentimos que poderia valer a pena considerar essa peculiaridade como uma fonte complementar de dados de avaliação.
  
  Tomando Rails comunidade OSS e projetos como quadro, o presente artigo discute o papel da
  \emph{boas práticas} melhores práticas na medição de qualidade de software e descreve os estudos realizados, 
  a construção de um novo código de ferramenta de análises que
  permitirá aos usuários fazer melhores escolhas, sobre o que o software de usar e ajudar os desenvolvedores, 
  para melhorar seu software.
}
