\thispagestyle{plain}
\chapter*{Resumo}\label{chap:resumo}


{\it
Este documento apresenta uma tese de mestrado em Ciência da Computação, na área de \ textit {Programa Compreensão e análise estática de código}.

Este trabalho (preparação da tese e escrita) é um componente do segundo ano do mestrado nova (segundo ciclo de Bolonha), e será realizado na Universidade do Minho em Braga, Portugal.

 

  Milhares de software de fonte aberta (OOS) projetos estão disponíveis para a colaboração em plataformas como o Github ou Sourceforge.
  No entanto, como o software tradicional, projetos OOS têm diferentes níveis de qualidade.
  O desenvolvedor ou o usuário final, precisa saber a qualidade de um determinado projeto antes de iniciar a colaboração
  ou na sua utilização --- eles podem, naturalmente, para confiar no pacote antes de tomar uma decisão.
  No contexto da OSS, Confiabilidade é uma preocupação muito mais sensível, principalmente os usuários finais geralmente preferem pagar para
  software proprietário, para se sentir mais confiantes na qualidade do pacote.
  Projetos de OSS podem ser avaliados como os pacotes de software tradicionais que utilizam as métricas de software bem conhecidos.
  Neste trabalho, queremos ir mais longe e propõem um processo de grão mais fino para fazer análise de qualidade tal,
  ajustados precisamente para este ambiente de desenvolvimento único.
  Como é sabido, ao longo dos últimos anos, as comunidades de código aberto têm criado suas próprias normas e \ emph {} melhores práticas.
  No entanto, as métricas de software clássicos não levam em conta os \ emph {} melhores práticas
  estabelecido pela comunidade.
  Nós sentimos que poderia valer a pena considerar essa peculiaridade como uma fonte complementar de dados de avaliação.
  Tomando o Ruby OSS comunidade e projetos como quadro, o presente artigo discute o papel da
  \ Emph {} melhores práticas para medir a qualidade do software.

  Milhares de projetos de software livre estão disponíveis para a colaboração em plataformas como o Github ou Sourceforge.
  No entanto, não há informações sistemáticas sobre a qualidade desses projetos.
  
  Poucos usuários têm a capacidade de avaliar a qualidade de um projeto, olhando para o código-fonte.
  Uma aplicação, capaz de realizar a análise automática de um pacote de software e para gerar uma visão geral elevado nível de código,
  permitiria que os usuários façam as melhores escolhas, sobre qual software usar e ajudar os desenvolvedores,
  dando dicas sobre como melhorar seu software.
  
  Este artigo discute o que deve ser levado em conta na elaboração de tal aplicação e quais os benefícios podem ser esperados.

}
