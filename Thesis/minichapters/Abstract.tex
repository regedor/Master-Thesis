\thispagestyle{plain}
\chapter*{Abstract}\label{chap:abstract}

{\it
This document presents a master thesis in Computer Science, in the area of \textit{Program Comprehension and Static code analysis}.

This work (thesis preparation and writing) is a component of the second year of the new Masters degree (second cycle of Bologna), and will be achieved in University of Minho at Braga, Portugal. 

 

  Thousands of open source software (OOS) projects are available for collaboration in platforms like Github or Sourceforge.
  However, like traditional software, OOS projects have different quality levels.
  The developer, or the end-user, need to know the quality of a given project before starting the collaboration
  or its usage---they might of course to trust in the package before taking a decision.
  In the context of OSS, trustability is a much more sensible concern; mainly end-users usually prefer to pay for
  proprietary software, to feel more confident in the package quality.
  OSS projects can be assessed like traditional software packages using the well known software metrics.
  In this paper we want to go further and propose a finer grain process to do such quality analysis,
  precisely tuned for this unique development environment.
  As it is known, along the last years, open source communities have created their own standards and \emph{best practices}.
  Nevertheless, the classic software metrics do not take into account the \emph{best practices}
  established by the community.
  We feel that it could be worthwhile to consider this peculiarity as a complementary source of assessment data.
  Taking Ruby OSS community and projects as framework, this paper discusses the role of
  \emph{best practices} in measuring software quality.

  Thousands of open source software projects are available for collaboration in platforms like Github or Sourceforge.
  However, there is no systematic information about the quality of those projects.
  
  Few users have the ability to assess the quality of a project by looking at source code. 
  An application, able to perform automatic analysis of a software package and to generate a hight level overview of the code,
  would enable users to make better choices, about what software to use and help developers,
  by giving hints on how to improve their software.
  
  This paper discusses what should be taken into account when developing such application and what benefits can be expected.

}
