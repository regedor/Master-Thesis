\thispagestyle{plain}
\chapter*{Abstract}\label{chap:abstract}

{\it
  This document presents a master thesis in Computer Science, in the area of \textit{Program Comprehension and Static Code Analysis}.
  This work (thesis preparation and writing) is a component of the second year of the Masters degree,
  that will be achieved in University of Minho at Braga, Portugal. 

  Thousands of open source software (OSS) projects are available for collaboration in platforms like Github or Sourceforge.
  However, like traditional software, OSS projects have different quality levels.
  The developer, or the end-user, needs to know the quality of a given project before starting the collaboration
  or its usage---they may trust in the package before making a decision.

  In the context of OSS, trustability is a much more sensible concern; mainly end-users usually prefer to pay for
  proprietary software in order to feel more confident in the package quality.
  OSS projects can be assessed like traditional software packages using the well known software metrics.

  In this document we want to go further and propose a finer grain process to do such quality analysis,
  precisely tuned for this unique development environment.
  As it is known, along the last years, open source communities have created their own standards and \emph{best practices}.
  Nevertheless, the classic software metrics do not take into account the \emph{best practices}
  established by the community.
  The notion that it could be worthwhile to consider this peculiarity as a complementary source of assessment data 
  was the essence of this thesis.
  
  Taking Rails OSS community and its projects as framework, this document discusses the role of
  \emph{best practices} in measuring software quality and describes the studies carried out, to build a new code analysis tool that 
  will enable users to make better choices about what software to use and help developers to improve their software.
}
