\documentclass[14pt,compress]{beamer}
\mode<presentation>{\usetheme{Copenhagen}} 
\usepackage{tikz}
\usepackage{verbatim}
\usepackage{graphicx}
\usepackage[english,portuges]{babel}
\usepackage[utf8]{inputenc}
\usepackage{fancyvrb}
\usepackage{url}
\usepackage{multicol}
\usepackage{color}
\usepackage{listings}
\usepackage[T1]{fontenc}
\usepackage{ae}
\usepackage{ulem}
\usepackage{float}
\usepackage{multirow}
\usepackage{threeparttable}

\lstset{language=[LaTeX]TeX}

\newcommand{\mr}[2]{\multirow{#1}{*}{#2}}


\defbeamertemplate*{headline}{Copenhagen theme}
{%
  \leavevmode%
  \begin{beamercolorbox}[wd=.5\paperwidth,ht=2.5ex,dp=1.125ex,right]{section in head/foot}%
  \usebeamerfont{section in head/foot}\insertsectionhead\hspace*{2ex}
  \end{beamercolorbox}%
  \begin{beamercolorbox}[wd=.5\paperwidth,ht=2.5ex,dp=1.125ex,left]{subsection in head/foot}%
  \usebeamerfont{subsection in head/foot}\hspace*{2ex}\insertsubsectionhead
  \end{beamercolorbox}%
}


\defbeamertemplate*{footline}{ntnu theme}
{%
  \leavevmode%
  \hbox{\begin{beamercolorbox}[wd=.5\paperwidth,ht=2.5ex,dp=1.125ex,leftskip=.3cm,rightskip=.3cm]{author in head/foot}%
    \usebeamerfont{author in head/foot}%\insertframenumber{}
	\hfill\insertshortauthor
  \end{beamercolorbox}%
  \begin{beamercolorbox}[wd=.5\paperwidth,ht=2.5ex,dp=1.125ex,leftskip=.3cm,rightskip=.3cm plus1fil]{title in head/foot}%
    \usebeamerfont{title in head/foot}\insertshorttitle
  \end{beamercolorbox}}%
  \vskip0pt%
}


\definecolor{myblue}{rgb}{0.4,0.5,1}
\setbeamercolor{structure}{fg=myblue}
\setbeamertemplate{navigation symbols}{}


\title[Analysing and Measuring Open Source Projects]{Analysing and Measuring Open Source Projects}
\institute{MI-STAR 2011, UCE15, Universidade do Minho}
\author[Miguel Regedor]{\small{Miguel Regedor, \texttt{miguelregedor@gmail.com} }}
\date{31 de Janeiro de 2011}


\begin{document}

\begin{frame}[plain]
	\begin{center}
	%\pgfimage[width=0.30\textwidth]{img/bglogo}
	\maketitle
	\end{center}
\end{frame}


\begin{frame}{Tabel of Contents}\small{\tableofcontents[hideallsubsections]}\end{frame}

%%%%%%%%%%%%%%%%%%%%%%%%%%%%%%%%%%%%%%
%% Section Introduction
\section{Context/Motivation}
%\begin{frame}{Tabel of Contents}\small{\tableofcontents[currentsection,hideallsubsections]}\end{frame}
\frame{\frametitle{Open Source Software}
  Nowadays, Open Source Software is disseminated
  \begin{block}{Open Source Projects}
    \begin{itemize}
    \item Android
    \item Apache
    \item Mozilla Firefox
    \item Open Office
    \item Ubuntu 
    \end{itemize}
  \end{block}
}

\frame{\frametitle{Open Source Software}
  \begin{itemize}
  \item Thousands of OSS packages can be found online and free to download
\pause
  \item OSS is not only used by computer specialists
\pause
  \item Open Source can be now considered the largest software industry in the world
  \end{itemize}
\pause
  \begin{block}{~}
    Measuring the savings that people are making in licence fees, the open source industry is worth 60 billion dollars.
  \end{block}
}

\frame{\frametitle{Development Process}
  Open Source communities work in a \textit{bazaar style}
\pause
  \begin{block}{Traditional Software Development Process}
    \begin{itemize}
    \item Similar to building cathedrals
    \item Few specialized individuals working in isolation
    \end{itemize}
  \end{block}
  \begin{block}{Open Source Development Process}
    \begin{itemize}
    \item Chaotic way
    \item Resemble a great babbling \textit{bazaar}.
    \end{itemize}
  \end{block}
}

\frame{\frametitle{Quality}
  Can software that is developed in such chaotic way be trusted as a high quality product?
\pause
  \begin{block}{~}
  The shock is that in fact the \textit{bazaar style} seemed to work
  \end{block}
\pause
  However, how can the quality of an open source software project be measured?
}

\frame{\frametitle{Meusuring Open Source Sofware quality}
    \begin{itemize}
    \item A OSSP is built up from hundreds, sometimes thousands, of files
\pause
    \item To analyse manually a software project is a very hard and time consuming task
    \end{itemize}
\pause
  \begin{block}{With that in mind}
    A system capable of automatically analysing and producing reports would enable users to make better choices,
    and developers to further improve their software.
  \end{block}
}

\frame{\frametitle{Open Source Project Hosting Web Sites}
  \tiny{\begin{table}
  \begin{tabular}{|c|c|c|c|c|c|} \hline 
  Name & Established & Available VCS & Users & Projects & Alexa rank \\\hline
  \mr{5}{SourceForge}      &\mr{5}{1999}  &CVS            &\mr{5}{2,000,000}   &\mr{5}{236,319}          &\mr{5}{136}            \\
                           &              &SVN            &                    &                         &                       \\
                           &              &Bazar          &                    &                         &                       \\
                           &              &GIT            &                    &                         &                       \\
                           &              &Mercurial      &                    &                         &                       \\\hline 
  \mr{1}{GitHub}           &\mr{1}{2008}  &GIT            &\mr{1}{505,000}     &\mr{1}{1,516,000}        &\mr{1}{742}            \\\hline 
  \mr{2}{Google Code }     &\mr{2}{2006}  &SVN            &\mr{2}{?}           &\mr{2}{250,000}          &\mr{2}{900}            \\
                           &              &Mercurial      &                    &                         &                       \\\hline 
  \mr{3}{Code Plex}        &\mr{3}{2006}  &SVN            &\mr{3}{151,782}     &\mr{3}{15.955}           &\mr{3}{2,343}          \\
                           &              &Microsoft TFS  &                    &                         &                       \\
                           &              &Mercurial      &                    &                         &                       \\\hline 
  \mr{2}{Assembla}         &\mr{2}{2006}  &SVN            &\mr{2}{180,000}     &\mr{2}{60,000}           &\mr{2}{6,628}          \\
                           &              &GIT            &                    &                         &                       \\\hline 
  \mr{1}{Launchpad}        &\mr{1}{2005}  &Bazar          &\mr{1}{1,140,345}   &\mr{1}{19,016}           &\mr{1}{12,466}         \\\hline 
  \mr{4}{BerliOS}          &\mr{4}{2000}  &CVS            &\mr{4}{47,285}      &\mr{4}{5,448}            &\mr{4}{17,299}         \\
                           &              &SVN            &                    &                         &                       \\
                           &              &GIT            &                    &                         &                       \\
                           &              &Mercurial      &                    &                         &                       \\\hline 
  \mr{1}{Bitbucket}        &\mr{1}{2008}  &Mercurial      &\mr{1}{51,600}      &\mr{1}{27,769}           &\mr{1}{12,047}         \\\hline 
  \mr{1}{Gitorious}        &\mr{1}{2008}  &GIT            &\mr{1}{?}           &\mr{1}{8,336}            &\mr{1}{28,531}         \\\hline 
  \end{tabular}
  \end{table}}
}



%%%%%%%%%%%%%%%%%%%%%%%%%%%%%%%%%%%%%%
%% Assessing Open Source Software
\section{Assessing Open Source Software}
%\begin{frame}{Tabel of Contents}\small{\tableofcontents[currentsection,hideallsubsections]}\end{frame}
\frame{\frametitle{Assessing Open Source Software}
  \small{\begin{block}{IISO/IEC 912 software quality attributes}
    \begin{itemize}
    \item {\bf Functionality   } (meeting of the functional requirements)
    \item {\bf Reliability     } (fault tolerance and recoverability)
    \item {\bf Usability       } (effort to learn, and operate the software)
    \item {\bf Efficiency      } (performance and resource consumption)
    \item {\bf Maintainability } (effort to modify the software)
    \item {\bf Portability     } (effort to transfer to another environment)
    \end{itemize}
  \end{block}}
}


%%%%%%%%%%%%%%%%%%%%%%%%%%%%%%%%%%%%%%
%% Software Metrics
\section{Software Metrics}
%\begin{frame}{Tabel of Contents}\small{\tableofcontents[currentsection,hideallsubsections]}\end{frame}
\frame{\frametitle{Software Metrics}
  \begin{itemize}
  \item Lines of Code
\pause
  \item Cyclomatic Complexity
\pause
  \item Fan-In and Fan-Out
\pause
  \item Object-Oriented Metrics
\pause
  \item Coding Standards?
  \end{itemize}
}


%%%%%%%%%%%%%%%%%%%%%%%%%%%%%%%%%%%%%%
%% Available Tools
\section{Available Tools}
%\begin{frame}{Tabel of Contents}\small{\tableofcontents[currentsection,hideallsubsections]}\end{frame}
\frame{\frametitle{Available Tools}
  \small{\begin{itemize}
  \item {\bf FindBugs}    find "bugs" in Java Programs
  \item {\bf FxCop}       reports information about the assemblies
  \item {\bf PMD}         scans Java source code for potential problems
  \item {\bf PreFast}     identifies defects in C/C++ programs 
  \item {\bf RATS}        finds C/C++, Perl and Python security problems 
  \item {\bf SWAAT}       scans Java, JSP, ASP .Net and PHP
  \item {\bf Flawfinder}  scans C and C++
  \end{itemize}}
}


%%%%%%%%%%%%%%%%%%%%%%%%%%%%%%%%%%%%%%
%% Future Work
\section{Conclusion: Summary and Future Work}
%\begin{frame}{Tabel of Contents}\small{\tableofcontents[currentsection,hideallsubsections]}\end{frame}
\frame{\frametitle{Conclusion: Summary and Future Work}
  \begin{block}{Karatsuba Algorithm (1960)} Divide and Conquer \end{block}
}

\frame{\frametitle{Future Work}
  \begin{enumerate}
  \item Chose a language: Ruby
  \pause
  \item Select consensual good and bad projects
  \pause
  \item Find patterns
  \pause
  \item Create a systematic way to measure it
  \pause
  \item Continue the work
  \end{enumerate}

}

\frame{\frametitle{Analysing and Measuring Open Source Projects}
  \begin{block}{~} Questions? \end{block}
}


\end{document}


