\documentclass{beamer}

\usepackage{color}
\usepackage{pgf}
\usepackage{amsmath,amssymb,amsfonts}
\usepackage[english]{babel}
\usepackage[utf8]{inputenc}
\usepackage{url}
\usepackage{listings}
\usepackage{eurosym}
\usepackage{float}
\usepackage{multicol}
\usepackage{multirow}
\usepackage{threeparttable}

\newcommand{\mr}[2]{\multirow{#1}{*}{#2}}

\definecolor{darkgreen}{rgb}{0,0.7,0}
\definecolor{darkred}{RGB}{177,51,60}

\usetheme{Warsaw}
\setbeamercolor{frametitle}{bg=darkred}
\setbeamercolor{block title}{bg=darkred}
\setbeamercolor{title}{bg=darkred}
\setbeamertemplate{headline}{}
\setbeamercolor{structure}{fg=darkred}
\setbeamertemplate{navigation symbols}{}
\setbeamertemplate{footline}
{
  \makebox[1\paperwidth][c] % full-width center-aligned box

  \makebox[1\paperwidth][r] % full-width right-aligned box
  {\insertframenumber{} / \inserttotalframenumber \hspace{5px}}

  \vspace{5px}
}
\setbeamertemplate{itemize items}[default]
\setbeamertemplate{enumerate items}[default]


\title[The Role of Best Practices in Assessing Software Quality]{The Role of Best Practices in Assessing Software Quality}
\institute{University of Minho}
\author[Miguel Regedor]{\textbf{Miguel Regedor}, Pedro Rangel Henriques, Daniela da Cruz}
\date{5th International Workshop on Foundations and Techniques for \\ Open Source Software Certification \\ Montevideo, Uruguay, 14-15 November 2011}


\begin{document}
\begin{frame}[plain]
  \begin{center}
    %\pgfimage[width=0.30\textwidth]{img/bglogo}
    \maketitle
  \end{center}
\end{frame}



%%%%%%%%%%%%%%%%%%%%%%%%%%%%%%%%%%%%%%
%
%\begin{frame}{Tabel of Contents}\small{\tableofcontents[currentsection,hideallsubsections]}\end{frame}


%\section{Context/Motivation}
\frame{\frametitle{Context: OSS}
  Nowadays, Open Source Software is disseminated as a natural practice among SW developers
  \begin{block}{Open Source Projects}
    \begin{itemize}
    \item Android
    \item Apache
    \item Mozilla Firefox
    \item Open Office
    \item Ubuntu
    \end{itemize}
  \end{block}
}

%\section{Context/Motivation}
\frame{\frametitle{Context: OSS}
  \begin{itemize}
  \item Thousands of Open Source SW packages can be found online and free to download
  \item Open Source Project Hosting Web Sites
  \end{itemize}

  \tiny{\begin{table}
  \begin{tabular}{|c|c|c|c|c|c|} \hline
  Name & Established & Available VCS & Users & Projects & Alexa rank \\\hline
  \mr{5}{SourceForge}      &\mr{5}{1999}  &CVS            &\mr{5}{2,000,000}   &\mr{5}{236,319}          &\mr{5}{136}            \\
                           &              &SVN            &                    &                         &                       \\
                           &              &Bazar          &                    &                         &                       \\
                           &              &GIT            &                    &                         &                       \\
                           &              &Mercurial      &                    &                         &                       \\\hline
  \mr{1}{\textbf{GitHub}}           &\mr{1}{2008}  &GIT            &\mr{1}{505,000}     &\mr{1}{1,516,000}        &\mr{1}{742}            \\\hline
  \mr{2}{Google Code }     &\mr{2}{2006}  &SVN            &\mr{2}{?}           &\mr{2}{250,000}          &\mr{2}{900}            \\
                           &              &Mercurial      &                    &                         &                       \\\hline
  \mr{3}{Code Plex}        &\mr{3}{2006}  &SVN            &\mr{3}{151,782}     &\mr{3}{15.955}           &\mr{3}{2,343}          \\
                           &              &Microsoft TFS  &                    &                         &                       \\
                           &              &Mercurial      &                    &                         &                       \\\hline
  \mr{2}{Assembla}         &\mr{2}{2006}  &SVN            &\mr{2}{180,000}     &\mr{2}{60,000}           &\mr{2}{6,628}          \\
                           &              &GIT            &                    &                         &                       \\\hline
  \mr{1}{Launchpad}        &\mr{1}{2005}  &Bazar          &\mr{1}{1,140,345}   &\mr{1}{19,016}           &\mr{1}{12,466}         \\\hline
  \mr{4}{BerliOS}          &\mr{4}{2000}  &CVS            &\mr{4}{47,285}      &\mr{4}{5,448}            &\mr{4}{17,299}         \\
                           &              &SVN            &                    &                         &                       \\
                           &              &GIT            &                    &                         &                       \\
                           &              &Mercurial      &                    &                         &                       \\\hline
  \mr{1}{Bitbucket}        &\mr{1}{2008}  &Mercurial      &\mr{1}{51,600}      &\mr{1}{27,769}           &\mr{1}{12,047}         \\\hline
  \mr{1}{Gitorious}        &\mr{1}{2008}  &GIT            &\mr{1}{?}           &\mr{1}{8,336}            &\mr{1}{28,531}         \\\hline
  \end{tabular}
  \end{table}}
}


%\section{Context/Motivation}%\section{Context/Motivation}
\frame{\frametitle{Context: OSS quality}
%  Can Open Source Software be trusted as a high quality product?
%
We need to be confident in a SW Project developed under this framework!\\

So, we search for an answer to the question:\\
\begin{center}
\emph{How can the quality of an OSS Project be measured?}
\end{center}
}

\frame{\frametitle{Motivation}

In this context, we propose to consider the role of \emph{best practices in SW development}\\

\begin{center}to improve  the quality assessment of an OSS Project\end{center}
}

%\begin{frame}{Topics}
%\thispagestyle{empty}
%\small{\tableofcontents[hideallsubsections]}
%\end{frame}

\section{Open Source Software}
\frame{\frametitle{Open Source Software}
  \begin{itemize}
  \item Open Source Software(OSS) packages can be found in public web-based repositories
  \item OSS is not only used by computer specialists
  \item Open Source can be now considered the largest software industry in the world involving billions of dollars.
  \end{itemize}
}

\frame{\frametitle{OSS: Development Process}
\begin{center}Development Process\end{center}
}

\frame{\frametitle{OSS: Development Process}
  Open Source communities work in a \textit{bazaar style}
  \begin{block}{Traditional Software Development Process}
    \begin{itemize}
    \item Similar to building cathedrals; like an orchestra.
    \item Few specialized individuals working in controlled and closed environment
    \end{itemize}
  \end{block}
  \begin{block}{Open Source Development Process}
    \begin{itemize}
    \item Chaotic way
    \item Resemble a great babbling \textit{bazaar}.
    \end{itemize}
  \end{block}
}

\frame{\frametitle{OSS: Quality}
  Can software that is developed in such chaotic way be \textbf{trusted} as a high quality product?
  \begin{block}{~}
  The shock is that in fact the \textit{bazaar style} seems to work !
  \end{block}
  However, it is mandatory to create a method to measure the quality of the OSS Projects (OSSP)
}

%%%%%%%%%%%%%%%%%%%%%%%%%%%%%%%%%%%%%%
%% Assessing Open Source Software
\section{Assessing Open Source Software}
\frame{\frametitle{Measuring OSS Quality}
  An obvious answer is to apply traditional SW-Metrics to OSSPs.\\
  This implies 
  \begin{itemize}
    \item to elect a set of SW characteristics that impact in the product quality
    \item to define a formula to associate with each one a numerical value
    \item to learn how to read those values to make statements about the quality
  \end{itemize}
  However, we want to go further, and measure how programmers comply with the best practices.
}


\frame{\frametitle{(Traditional) Software Characteristics}
  \small{\begin{block}{ISO/IEC 912 software quality attributes}
    \begin{itemize}
    \item {\bf Functionality   } (meeting of the functional requirements)
    \item {\bf Reliability     } (fault tolerance and recoverability)
    \item {\bf Usability       } (effort to learn, and operate the software)
    \item {\bf Efficiency      } (performance and resource consumption)
    \item {\bf Maintainability } (effort to modify the software)
    \item {\bf Portability     } (effort to transfer to another environment)
    \end{itemize}
  \end{block}}
}


%%%%%%%%%%%%%%%%%%%%%%%%%%%%%%%%%%%%%%
%% Software Metrics
%\section{Software Metrics}

\frame{\frametitle{(Traditional) Software Metrics}
  \begin{itemize}
  \item Lines of Code
  \item Number of Components (packages, procedures, ...)
  \item Size and complexity of Data Structure 
  \item Cyclomatic Complexity (control flow and structural  complexity)
  \item Fan-In and Fan-Out (component dependencies)
  \item Object-Oriented Metrics (number of classes, methods, attributes, dependencies, etc.)
%  \item Coding Standards?
  \end{itemize}
}

\section{Best Practices}
\frame{\frametitle{Best Practices in OSS development}
  \begin{block}{~}
  Best Practices are not formal rules but instead a set of \emph{recommendations} that are spread
  through the community and everybody does it that way.
  \end{block}
  Best Practices provide a set of pragmatic suggestions/tips on code writing in order to improve the program
 Readability, Maintainability, Portability, and also its Usability,  and Efficiency.
   
  Furthermore, best practices discourage the use of bad patterns, poor error checking, etc,
}

\frame{\frametitle{The Question}
\begin{center}Are best practices and the project quality truly related?\end{center}
}

\end{document}


